\chapter{经典论文}
\section{YouTube DNN}

文献 \citerb{2016-Youtube-DNN} 介绍了 DNN 在 YouTube 视频推荐上的应用。

\subsection{推荐模型整体结构}
YouTube DNN 推荐模型的整体结构如图~\ref{fig:youtube-dnn-structure}~所示,包括召
回和排序两级模型:

\begin{figure}[ht]
  \centering
  \includegraphics[width=0.8\textwidth]{images/YouTube/YouTube-DNN-推荐模型结构.pdf}
  \caption{YouTube DNN 推荐模型结构}
  \label{fig:youtube-dnn-structure}
\end{figure}

\begin{itemize}
  \item 召回模型:召回模型主要利用用户的历史信息,从视频库中的几百万视频中挑出几
    百个用户感兴趣的视频。
  \item 排序模型:排序模型综合利用用户的历史信息和视频特征,将召回模型的输出和其
    他来源的视频进行排序,最终输出十个左右的候选视频。
\end{itemize}

\subsection{召回模型结构}
召回模型将挑出用户感兴趣的视频建模成一个多分类问题:

\begin{equation}
  P(w_t=i|U, C) = \frac{e^{v_i u}}{\sum_{j \in V} e^{v_j u}}
\end{equation}

其中 $u$ 和 $v$ 分别为用户和视频的 embedding。模型结构如图~\ref{fig:youtube-dnn-retrieval-structure}~所示:

\begin{figure}[ht]
  \centering
  \includegraphics[width=0.8\textwidth]{images/YouTube/YouTube-DNN-召回模型结构.pdf}
  \caption{YouTube DNN 召回模型结构}
  \label{fig:youtube-dnn-retrieval-structure}
\end{figure}

召回模型的目标是学习如何将用户的观看历史、搜索历史和个人信息映射为用户 embedding,
以区分不同用户感兴趣的视频。

\paragraph{模型输入 feature}
\begin{itemize}
  \item 用户观看历史:首先将变长稀疏的 video ID 观看历史映射为 embedding,再将
    embedding 做平均得到固定长度的输入。
  \item 用户搜索历史:处理方法与用户观看历史类似。
  \item 用户个人信息:包括用户地理信息和设备的 embedding 以及归一化的用户性别、年
    龄、登录状态等。
  \item example age:\color{red}{待补充}
\end{itemize}
\paragraph{正负样本} 用户完整看完一个视频作为正样本。训练样本包含所有观看记录,包
括其他网站嵌入的 YouTube 视频。每个用户都生成相等数量的训练样本,以降低高活跃用户
的影响。
\paragraph{训练} 由于 softmax 的类别非常多(视频库中的视频数),训练时采用负类采
样 + 重要性加权的方式进行加速,不再同时计算所有类别的 loss。
\paragraph{部署} 部署时由于有时延要求,使用近似最近邻方法找到 top N 个视频。
\paragraph{评测} 模型使用的训练样本包括 $t$ 时间前的样本,预测 $t$ 时刻的观看,
避免随机留一法导致的信息泄露。

\subsection{排序模型结构}
排序模型的目标是根据用户的期望观看时长对视频排序,其模型结构如图~\ref{fig:youtube-dnn-structure}~所示:

\begin{figure}[ht]
  \centering
  \includegraphics[width=0.8\textwidth]{images/YouTube/YouTube-DNN-排序模型结构.pdf}
  \caption{YouTube DNN 排序模型结构}
  \label{fig:youtube-dnn-ranking-structure}
\end{figure}

\paragraph{模型输入 feature}
\begin{itemize}
  \item 视频 ID:综合展示的视频和用户观看的视频生成 embedding,取平均后作为输入。
    计算 embedding 时只考虑 top N 个。
  \item 语言:综合用户语言和视频语言生成 embedding。
  \item 距上次观看的时间:先利用累积分布函数做归一化,同时引入非线性的平方项和开
    方项,增强模型表达能力。
  \item 之前展示的视频数:与距上次观看的时间处理类似。
\end{itemize}

\paragraph{特征工程}
\begin{itemize}
  \item 用户兴趣:用户在该频道看过的视频数、上次观看该频道的时间
  \item 召回模型信息:视频来源、召回模型输出的分数
  \item 视频展示频率:降低已展示未点击视频的出现频率
\end{itemize}

\paragraph{训练和部署} 排序模型的数学形式为带权重的逻辑回归,其中正样本为用户点击
展示的视频,其权重为观看时长;负样本为用户未点击展示的视频,其权重为 1。部署时,
使用 $e^{x}$,其中 $x$ 为逻辑回归模型的输出,$e^x$ 为每个视频的用户期望观看时长。

%%% Local Variables:
%%% TeX-master: "../master"
%%% End:
