\part{机器学习和深度学习基础}
\chapter{基本概念}
\section{激活函数}

\subsection{Sigmoid 函数}
\label{subsec:Sigmoid}

Sigmoid 函数及其导数的形式为:
\begin{align}
  \label{equ:sigmoid}
  \sigma(x) & = \frac{1}{1 + e^{-x}} \\
  \label{equ:sigmoid-d}
  \sigma'(x) & = \sigma(x) (1-\sigma(x))
\end{align}

% \begin{figure}[ht]
%   \centering
%   \includegraphics[width=0.8\textwidth]{images/机器学习和深度学习基础/基本概念/Sigmoid.png}
%   \caption{Sigmoid 函数及其导数的图像}
%   \label{fig:sigmoid}
% \end{figure}

\begin{itemize}
  \item 图~\ref{fig:sigmoid}~中蓝色曲线为 sigmoid 函数,其形状与字母 S 类似,可以
    将 $-\infty$ 到 $\infty$ 的输入单调映射到 0 到 1 之间。
  \item 图~\ref{fig:sigmoid}~中红色曲线为 sigmoid 函数的导数,在输入值较大或较小时,
    该值趋于 0,因此 sigmoid 函数作为激活函数时,可能出现梯度消失,网络无法正常训
    练。
\end{itemize}

%%% Local Variables:
%%% TeX-master: "../master"
%%% End:
