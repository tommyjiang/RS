\chapter{经典论文}
\section{YouTube DNN}

文献 \citerb{2016-Youtube-DNN} 介绍了 DNN 在 YouTube 视频推荐上的应用。

\subsection{推荐模型整体结构}
YouTube DNN 推荐模型的整体结构如图~\ref{fig:youtube-dnn-structure}~所示,包括召
回和排序两级模型:

\begin{figure}[ht]
  \centering
  \includegraphics[width=0.8\textwidth]{images/YouTube/YouTube-DNN-推荐模型结构.pdf}
  \caption{YouTube DNN 推荐模型结构}
  \label{fig:youtube-dnn-structure}
\end{figure}

\begin{itemize}
  \item 召回模型:召回模型主要利用用户的历史信息,从视频库中的几百万视频中挑出几
    百个用户感兴趣的视频。
  \item 排序模型:排序模型综合利用用户的历史信息和视频特征,将召回模型的输出和其
    他来源的视频进行排序,最终输出十个左右的候选视频。
\end{itemize}

\subsection{召回模型结构}
召回模型将挑出用户感兴趣的视频建模成一个多分类问题:

\begin{equation}
  P(w_t=i|U, C) = \frac{e^{v_i u}}{\sum_{j \in V} e^{v_j u}}
\end{equation}

\subsection{排序模型结构}

%%% Local Variables:
%%% TeX-master: "../master"
%%% End:
